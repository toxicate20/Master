% abstract

\textit{Dose painting} is a new way to incorporate imaging of tumour biology into treatment planning to improve the efficacy of radiation therapy.  Hypoxia has been identified as one of the major reasons for the failure of local tumour control in head and neck cancer. The lack of oxygen enhances DNA repair from radiation damage, which leads to higher radioresistance in such hypoxic tumour regions. Dose painting can help to overcome hypoxia by increasing the dose to account for higher radioresistance. This thesis will introduce a new way of dose painting using the biological effect (\textit{biological dose painting}) to overcome the treatment deficiencies created by hypoxia. Ten head and neck cancer patients with clinically approved plans from the Princess Margaret Hospital (Toronto, Ontario, Canada) have been used to apply biological dose painting under iso-toxic treatment planning to show feasibility of the new approach. Afterwards the deployed models are examined in terms of their model parameter uncertainties to assess the robustness of biological dose painting. This work shows, that biological dose painting is feasible and compensates for changes in radioresistance caused by hypoxia. All critical structures are spared according to clinical constraints while hypoxic volumes receive an appropriate dose to compensate for higher radioresistance.\\\\\itextit{Dose painting} ist eine neuartige Methode biologische Bildgebung in den Prozess der Bestrahlungsplanung einzubinden. Durch dieses Verfahren wird die Effizienz von Strahlungstherapien verbessert. Eine der h\"aufigsten Ursachen f\"ur schlechte Ergebnisse im Rahmen der Therapie von Kopf- und Halstumoren stellt die Hypoxie dar. Der Begriff der Hypoxie beschreibt die Mangelversorgung eines Gewebes mit Sauerstoff. Das Fehlen des Sauerstoffs im umgebenen Gewebe beg\"unstigt die Reparatur der zuvor durch Strahlung zerst\"orten Tumor-DNA, wodurch es zu einer erh\"ohten Radioresistenz kommen kann. Durch entsprechende Erh\"ohung der Strahlendosis kann dose painting dabei helfen, den Effekt der Hypoxie zu minimieren und somit das Therapieergebnis zu verbessern. Das Ziel dieser Arbeit ist die Einf\"uhrung des biologischen dose painting, bei dem der biologische Effekt Anwendung findet. Durch diese neuartige Methode sollen Behandlungm\"angel im hypoxischen Volumen ausgeglichen werden. Die Durchf\"uhrbarkeit und Effizient des biologischen dose painting wird in dieser Arbeit anhand klinisch zugelassener Patientendaten des Princess Margaret Hospital (Toronto, Ontario, Kanada) dargelegt. Ebenso werden die Ergebnisse der oben genannten Methode auf ihre Robustheit gegen\"uber Modellunsicherheiten gepr\"uft. Insgesamt verdeutlicht diese Arbeit die Anwendbarkeit des biologischen dose painting. Ebenso wird gezeigt, dass auch hypoxische Gewebe, welche eine erh\"ohte Radioresistenz aufweisen, mit dieser Methode behandelt werden k\"onnen.