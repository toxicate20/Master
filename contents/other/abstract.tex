% abstract

\textit{Dose painting} is a new way to incorporate imaging of tumour biology into treatment planning to improve the efficacy of radiation therapy.  Hypoxia has been identified as one of the major reasons for the failure of local tumour control in head and neck cancer. The lack of oxygen enhances DNA repair from radiation damage, which leads to higher radioresistance in such hypoxic tumour regions. Dose painting can help to overcome hypoxia by increasing the dose to account for higher radioresistance. This thesis will introduce a new way of dose painting using the biological effect (\textit{biological dose painting}) to overcome the treatment deficiencies created by hypoxia. Ten head and neck cancer patients with clinically approved plans from the Princess Margaret Hospital (Toronto, Ontario, Canada) have been used to apply biological dose painting under iso-toxic treatment planning to show feasibility of the new approach. Afterwards the deployed models are examined in terms of their model parameter uncertainties to assess the robustness of biological dose painting. This work shows, that biological dose painting is feasible and compensates for changes in radioresistance caused by hypoxia. All critical structures are spared according to clinical constraints while hypoxic volumes receive an appropriate dose to compensate for higher radioresistance.\\\\\textit{Dose painting} ist ein neuartiger Weg, biologische Bildgebung in den Prozess der Bestrahlungsplanung zu integriert um die Effizienz von Bestrahlungstherapie zu verbessern. Hypoxie ist eine der h\"aufigsten Gr\"unde f\"ur die fehlende Tumorkontrolle in Kopf und Hals Tumoren. Fehlender Sauerstoff in der Tumorumgebung erh\"oht die Chancen von DNA Reparatur durch Strahlungssch\"aden, was zu einer h\"oheren Radioresistenz in diesen Regionen f\"uhren kann. Dose painting kann dabei helfen, Hypoxie durch eine entsprechend h\"ohere Dosis zu \"uberwinden. Diese Arbeit f\"uhrt eine neue Methode f\"ur dose painting ein, die den biologischen Effekt (biologisches dose painting) benutzt, um Behandlungsm\"angel in hypoxischen Volumen auszugleichen. Zehn Kopf und Hals Patienten mit klinisch zugelassenen Pl\"anen vom Princess Margaret Hospital (Toronto, Ontario, Kanada) wurden f\"ur den Ansatz von biologischem dose painting unter isotoxischen Nebenbedingungen behandelt um die Machbarkeit dieser Methode zu zeigen. Danach wurden die angewandten Modelle bez\"uglich ihrer Robustheit gegen\"uber Modellunsicherheiten untersucht. Diese Arbeit zeigt, dass biologisches dose painting m\"oglich ist und hypoxische Gebiete mit h\"oherer Radioresistenz angemessen behandelt. Alle kritischen Strukturen wurden bez\"uglich der klinischen Auflagen verschont w\"ahrend hypoxische Volumen angemessene Dosis erhielten.