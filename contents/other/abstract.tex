% abstract

\textit{Dose painting} is a new way to combine biological imaging and radiation therapy to improve treatment planning. This approach incorporates various biological input parameters that increases survival in tumour cells.  Hypoxia has been identified as one of the major reasons for the failure of local tumour control in head and neck cancer. The lack of oxygen pronounces DNA repair from radiation damage, which leads to higher radioresistance in such hypoxic tumour regions. Dose painting can help to overcome hypoxia by increasing the dose to account for higher radioresistance. This thesis will introduce a new way of dose painting using the biological effect (\textit{biological dose painting}) to overcome the treatment deficiencies created by hypoxia. Ten head and neck cancer patients with clinically approved plans from the Princess Margaret Hospital (Toronto, Ontario, Canada) have been used to apply biological dose painting under iso-toxic treatment of critical structures to show feasibility of the new approach. Afterwards the deployed models are examined in terms of their model parameter uncertainties to assess the robustness of biological dose painting. This work shows, that biological dose painting is feasible and compensates for changes in radioresistance caused by hypoxia. All critical structures are spared according to clinical constraints while hypoxic volumes receive an appropriate dose to compensate for higher radioresistance.