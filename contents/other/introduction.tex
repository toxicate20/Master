The term \textit{dose painting} has been around since the year 2000 and has been coined by \textit{Ling et al.} \cite{pmid10837935}. It describes a method of merging biological imaging and treatment planning to increase local tumour control with the help of biological input parameters derived from tumour maps. Such maps are generated from functional images acquired through positron emission tomography (PET) or magnetic resonance imaging (MRI). The biological parameter which is mapped with PET depends on the radiopharmaceutical, which is administered into the patient. While FDG is used to show the metabolic state of the tissue \cite{pmid16841141}, FMISO is currently used as a hypoxia tracer in clinics to spot hypoxic volumes and incorporate them into treatment planning by increasing the dose in these volumes by an arbitrary dose value. One of the major drawbacks when linking the biological image information from PET to a biological tumour map is the ignorance toward to a proper interpretation of image data. Multiple approaches to dose painting have been proposed: Binary dose painting \cite{pmid20855118, pmid11240261, pmid17869020}, functional dose escalation \cite{pmid12587912, pmid21356478, pmid20643512, pmid18635895} and kinetic model dose painting \cite{pmid17448882}. With the exception of the latter, all dose painting approaches do not incorporate any biological motivation for the prescription function used to increase the dose in hypoxic volumes. \textit{Bentzen et al.} \cite{pmid19218733} showed how different prescription functions affect the resemblance of dose distributions. Generally the functional form of the prescription function that directly maps PET image information to dose escalations does not incorporate any biological motivation. Rather than depending on a direct prescription function, this thesis introduces a novel approach to dose painting called \textit{biological dose painting}. In contrast to the above mentioned concepts, the approach proposed in this thesis includes extensive clinical data to incorporate a realistic model for image interpretation of PET images to generate tumour maps for dose painting. The clinical data used to gauge the biological models introduced in this method are derived from multiple studies on oxygenation states of head and neck cancer from Eppendorf polarographic needle measurements as well as direct measurements of FMISO and FDG PET image intensities. To link the intensity of PET images and underlying oxygen values the conducted meta analysis has been connected with multiple studies that have investigated the correlations of FMISO PET and Eppendorf oxygen measurements \cite{pmid17598907, pmid12865184, pmid20831480}. This approach allows a study of the application of biological dose painting on head and neck patients with clinically approved plans (supplied by the Princess Margaret Hospital (Toronto, Ontario, Canada)) to assess feasibility and planning robustness with the applied models.\\The purpose of this thesis is to evaluate biological dose painting in head and neck radiotherapy and to assess its feasibility. All plans generated with biological dose painting conform to the dose constraints that were set in the clinical plans, while simultaneously increasing the dose to compensate for the radioresistance in hypoxic tumour volumes. Chapter \ref{chapter:2} deals with all physical principles and biological mechanisms that are involved in the cell survival of hypoxic tumour cells as well as the used models and derived parameters. Chapter \ref{chapter:3} will give a small overview on the above mentioned dose painting methods that have been proposed in the past years. Afterwards the novel approach of biological dose painting is described in detail. Chapter \ref{chapter:4} describes the results of the application of biological dose painting in head and neck patients, while chapter \ref{chapter:5} describes the efficacy of dose painting on uncertain biological tumour maps. The uncertainty of those maps is derived from the parameter errors of the deployed biological models. Chapter \ref{chapter:6} discusses different methods, that could increase accuracy and feasibility of dose painting.