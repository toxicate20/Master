% chapter 7 summary and conclusion
This thesis has shown the application of biological dose painting with clinical head and neck cancer patients. The main goal of this work was to show the feasibility of the method as well as the impact of model uncertainties on treatment planning. The first part of this chapter will discuss the results and summarize the findings of this thesis. The second part (marked with numbers in front of the paragraphs) are conclusions that arise from the deficiencies found with the biological dose painting approach.
\paragraph{Feasibility of dose painting} Treatment planning with biological dose painting was possible with the head and neck patients investigated in this thesis. The models deployed in this approach have been altered in such a way that they represent a realistic situation in the clinic. This was accomplished by a meta analysis and an in depth literature review for measurements on oxygen partial pressures as well as PET image intensities. The dose required in highly hypoxic tumour volumes with less than 2.5 mmHg oxygen partial pressure based on the 70 Gy prescription in conventional radiotherapy was 126 Gy. This value has been calculated from various clinical data together with the deployed models. The dose escalations from the biological models compensate for hypoxia which make biological dose painting more advantageous than a conventional treatment plan that does not incorporate hypoxia maps. The latter overestimates cell kill which then can lead to treatment failure. Usually the peak dose of 126 Gy was not achieve due to critical structures or the size of the hypoxic sub volumes. The latter deficiency was based on the construction of these sub volumes as they induced high dose gradients which are much more difficult to achieve than with a real PET image. This is due to the fact that those images are smooth and do not exhibit any steep intensity steps as it was the case with the PET images created for this approach. Therefore dose coverage and peak dose will be much easier to achieve with real PET image data. Critical structures that limited the dose delivery to the target volumes were the parotids, spinal cord, brainstem, mandible and the esophagus. Most critical structures could be spared by changing overlap priorities with respect to the treatment volumes. This is only needed, if the clinical target volumes show a larger overlap with the critical structures as the dose escalation to hypoxic sub volumes can go above 75 Gy which usually is the highest dose for any critical structure. In a few cases, feasibility of dose painting is limited if the critical structures are not disputable for such a high dose. Rather than overdosing a critical structure, the patient should not be treated with dose painting.
\paragraph{Robustness analysis} Planning robustness suffers from the large error on the model parameter $p_{50}$ while $m$ and $K$ are less influential on the hypoxia maps generated by the deployed models. If $p_{50} = 1$ mmHg, then most voxels are seen as highly hypoxic. This leads to a high dose escalation up to 150 Gy which is not feasible for any treatment planning scenario. For $p=11.8$ mmHg dose painting could be become even more feasible as the dose escalations from hypoxia maps require a maximum dose of 88.2 Gy. It should be mentioned that irregardless of the robustness of the model, biological dose painting does not perform worse than a conventional treatment plan as the dose escalations derived from HRF are never lower than the dose that was prescribed in the clinical plan. Therefore, biological dose painting should be used as soon as pO2 models become more sophisticated in terms of predicting hypoxia from PET images.
\paragraph{(1) Study with real PET images} The PET images used in this thesis were generated artificially. Although they are based on clinical findings from multiple literature sources, biological dose painting needs to be reevaluated with real PET images. As stated above, the approach used in this thesis to derive hypoxia maps introduces steep dose gradients which make a complete dose delivery much harder than with real PET images. This is due to the fact that real PET images have a much smoother intensity distribution which leads to a much smoother (and gradually increasing) dose escalation toward the hypoxic sub volumes.
\paragraph{(2) Intepretation of PET} The pO2 model used has a large error on the $p_{50}$ value. It would be desirable to reevaluate the model with a larger group of patients to get a better estimate on the model parameter. Another approach to this problem would be a meta study on the general correlation between FMISO and tumour oxygenation measured with Eppendorf polarographic needle electrodes. This investigation could yield a clear picture on how the relation between tracer uptake of FMISO and hypoxia is linked. With this information a general pO2 model based on those correlations could be formulated.
\paragraph{(3) Evaluation of tumour maps through reoxygenation} Reoxygenation occurs when tumour cells are damaged by ionizing radiation. This process can increase the oxygen partial pressure in specific volumes of the tumour permanently. This has a direct impact on the tumour map. As the treatment plan (consisting of 30 and more fractions) is currently calculated on an initial PET image, variations in tumour oxygenation are not incorporated in the biological dose painting approach. By generating multiple PET images during treatment planning and reevaluating the remaining fractions, dose could be adjusted to achieve lower total dose to the critical structures. Since PET imaging is quite expensive and reoxygenation only becomes significant after multiple fractions, such reevaluation could be done every 10-15 fractions. 
\paragraph{(4) Differentiation of hypoxia types} Chronic and acute hypoxia are both measured with FMISO as the retention mechanism of this radiopharmaceutical does not differentiate between those hypoxia types. Dose painting should only treat chronic hypoxia as it is prevalent during the course of the treatment. Acute hypoxia can only be evaluated with two PET images taken within a certain time frame. The application of the Wang model (cf. chapter \ref{chap:wang}) can help identify the fraction of acute hypoxia. By subtracting the effects of acute hypoxia from the PET image, new hypoxia maps can be generated that exhibit lower HRF and therefore lower dose escalations which can lead to even higher feasibility of biological dose painting.
\paragraph{(5) Use other tracers} The field of hypoxia tracers is evolving quickly. Multiple tracers have made their way to the market and are currently on the way to clinical studies. Most of them can hold the key to a better hypoxia image map which can help to accurately determine hypoxia through PET imaging. it is important to understand the relation of those new hypoxia tracers to the real hypoxia environment in the patient. An ideal tracer would have a high signal-to-background ratio as well as a clear correlation between the tracer retention and the prevalent hypoxia in the patient. 
\paragraph{(6) Higher PET resolutions} The major drawback of PET is its low resolution. As the scale of hypoxia on the cellular level is much smaller, PET is not able to map the microenvironment of hypoxic states of any tissue. PET is rather a mean value representation of the underlying biology. With the introduction of new and better techniques to improve PET image resolution, PET should be able to deliver more accurate images (and hypoxia maps) than currently possible.